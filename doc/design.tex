\section{Options for translating rules and .rudi files}

Naming convention: a label followed by a colon followed by an \emph{if} is
called \emph{rule} from now on.

There are two main options to translate:
\begin{itemize}
\item[A)] translate the whole project into one large Java class / method
  Problems with this approach
  \begin{itemize}
  \item The relation to the source code is hard to track: no modularity, one
    large blob
  \item The execution regime can not be changed except by changing the
    translation (no dynamic adaptation of execution)
  \end{itemize}
\item[B)] Create a class for each .rudi file and each top-level rule
  \begin{itemize}
  \item Clear structure that is isomorphic to the .rudi files
  \item Dynamic execution strategy is easier to imagine, albeit not really
    feasible (do we want/need it?)
  \item Variables on the top level of a file can be implemented by class
    fields and fully specified access, such as Introduction.special\_variable
  \end{itemize}
\end{itemize}

We're going for version B), assuming that most of the variables which have
non-local scope are state variables in the agent, which also might be
specified in special top-level files. These should also contain specifications
for general framework functions, i.e., the whole signatures including types
for the arguments and return types, to enable advanced type checking.

Similar files could be used for custom user state variables and functions,
i.e., for Quiz logic, custom knowledge bases, etc.

Embedded rules will be treated differently to avoid the overhead of handling
local scope and lifetime of variables.

\texttt{return} statements can have optional labels that indicate the exit
point / level, which may be local rule, top-level rule or file. A return
without label exits from the innermost scope.

Example for a top-level rule:

\begin{verbatim}
a:
if ( XXXX ) {
  x = child_27;
  b:
  if (YYYY) {
     x = child_3;
     c:
     if ( ZZZZ ) {
        ....
        if (....) return b:;
        ....
     } // c: ends
     ....
  } // b: ends
} // a: ends
\end{verbatim}

The labeled exits are implemented by setting flags in a bit mask that are
tested in the following to skip code not to be executed:
\newpage
Possible translation, (example starts at \texttt{return b:})
\begin{verbatim}
a:
if ( XXXX ) {
  x = child_27;
  b:
  if (YYYY) {
     x = child_3;
     c:
     if ( ZZZZ ) {
        ....
        if (....) {
           // return b:;
           returnMask |= b_mask;
        }
        if ((returnMask | (a_mask | b_mask | c_mask)) == 0) {
           ....
        }
     } // c: ends
     if ((returnMask | (a_mask | b_mask)) == 0) {
     ....
     }
  } // b: ends
  if ((returnMask | a_mask) == 0) {
  }
} // a: ends
\end{verbatim}



%%% Local Variables:
%%% mode: latex
%%% TeX-master: "master"
%%% End:
