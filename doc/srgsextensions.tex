\section{Extensions to the SRGS/VoiceXML Formalism} \label{sec:srgs}
\vonda comes with a simple NLU module that is based on the
SRGS\footnote{\url{ https://www.w3.org/TR/semantic-interpretation/}}
grammar format. SRGS grammars are so-called augmented context free
grammars, where the rules contain terminal and non-terminal symbols
that define the set of strings that are valid matches of the
\emph{language} that the grammar defines, but also semantic actions
that collect the \textrm{output} of the syntactic analysis in form of
possibly complex objects. Our implementaton of the SRGS
formalism\footnote{\url{https://github.com/bkiefer/srgs2xml}, for more details see the
  ReadMe and documentation there} provides some extensions to the
default grammar formalism, currently only in the semantics (action
part) of the rules, providing return values not only based on the names of non-terminal symbols, but also using relative positions of matched strings or tokens

\vspace*{1ex}
\begin{tabular}{ll}
\verb|$%n| & refers to the value returned by the rule $n$ positions before this tag\\
\verb|$$n| & refers to the the matched string $n$ positions before this tag\\
\end{tabular}
\vspace*{1ex}

Note that if \verb|$%n| or \verb|$$n| refers to an alternative, the result of the matched alternative is returned.

\texttt{n} always has to be greater than zero, since zero would be at the position
where the semantic element is that makes the tag reference, which does not make
sense. For determining \texttt{n}, you have to count \emph{every} grammar token,
including semantic tokens.

As an example, taken from the test cases of the \texttt{srgsparser}
module, we parse the sentence "I want a medium pizza with Mushrooms
please" with the grammar shown below, which will return a JSON object
of the form:

\vspace*{1ex}
\verb|{ "order": { "size": "medium", "topping": "mushrooms" } }|
\vspace*{1ex}

since the \verb|$%2| in the \texttt{size} rule returned the output of the \texttt{\$small $\vert$ \$medium $\vert$ \$large} alternative, and the \verb|$$1| returned the string matched in the \texttt{medium} rule immediately before the tag. The rest is done with the ordinary JSON semantics present in the standard formalism.

{\small%
\begin{verbatim}
#ABNF 1.0 UTF-8;

language en-EN;
root $order;
mode voice;
tag-format "semantics/1.0";

$politeness1 = [I want];
$politeness2 = [please];

$small = small {out = "$$1";};
$medium = medium {out = "$$1";};
$large = large {out = "big";};

$size = (($small | $medium | $large) pizza {out = $%2;})
        | (hot chili) { out="chili"; } ;

$topping = Salami {out = "salami";}
         | Ham {out = "ham";}
         | Mushrooms {out = "mushrooms";} ;

public $order =
  {out = new Object(); out.order = new Object;}
  $politeness1
  (
  [a] $size pizza {out.order.size = rules.size;}
  | [a] [pizza with] $topping {out.order.topping = rules.topping;}
  | [a] $size {out.order.size = rules.size;}
    with $topping {out.order.topping = rules.topping;}
  )
  $politeness2 ;
\end{verbatim}}

%%% Local Variables:
%%% mode: latex
%%% TeX-master: "userguide"
%%% End:
