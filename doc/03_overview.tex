\newcommand{\caret}{{\large\textbf{\textasciicircum}}}

\section{The \vonda Compiler}

The compiler turns the \vonda source code into Java source code using the
information in the ontology. Every source file becomes a Java class. The
generated code will not serve as an example of good programming practice, but a
lot of care has been taken in making it still readable and debuggable. The
compile process is separated into three stages: parsing and abstract syntax
tree building, type checking and inference, and code generation.

The \vonda compiler's internal knowledge about the program structure and the
RDF hierarchy takes care of transforming the RDF field accesses to reads from
and writes to the database. Beyond that, the type system, resolving the exact
Java, RDF or RDF collection type of arbitrary long field accesses,
automatically performs the necessary casts for the ontology accesses.

\section{\vonda's Architecture}

Figure~\ref{fig:architecture} shows the architecture of a runnable \vonda project.

\begin{figure}[htbp]
  \begin{center}
    \renewcommand{\sfdefault}{lmss}
    \tiny%
\begin{tikzpicture}[font=\bf\sffamily,
  box/.style={minimum width=4.5cm, minimum height=.35cm, draw=gray, text=black},
  gbox/.style={box, fill=lightgray},
  arr/.style={thick, -{Stealth}},
  larr/.style={thick, {Stealth}-},
  oarr/.style={thick, -{Stealth}},
  bbox/.style={box, fill=meddarkblue, text=ivory, draw},
  rbox/.style={node distance=0.3cm, font=\ttfamily},
  fbox/.style={node distance=4.2cm, font=\ttfamily},
  rlabel/.style={right, xshift=1mm, font=\sl},
  llabel/.style={left, xshift=-1mm, font=\sl}
]
\node[gbox](rmn) at (0,0.2) { Rule Module Class N };
\node[rbox, right= of rmn](rmnj){RuleModuleN.java};
\draw[larr] (rmnj) --  ++(1.5,0);
\node[fbox, right= of rmn](rmnr){RuleModuleN.rudi};
\draw[oarr] (rmnr) --  ++(-1.5,0);

\node[minimum width=4.5cm](ddd) at (0,-.2) { \normalsize ... };
\node[rbox, right= of ddd]{\small ...};
\node[fbox, right= of ddd]{\small ...};

\path (rmn) ++(0,-.75) node[gbox](rm1) { Rule Module Class 1 };
\node[rbox, right= of rm1](rm1j){RuleModule1.java};
\draw[larr] (rm1j) --  ++(1.5,0);
\node[fbox, right= of rm1](rm1r){RuleModule1.rudi};
\draw[oarr] (rm1r) --  ++(-1.5,0);

\path (rm1) ++(0,-.75) node[gbox](tlrc) { Top Level Rule Class };
\node[rbox, right= of tlrc](maj){MyAgent.java};
\draw[larr] (maj.east) ++(.35,0) --  ++(.72,0);
\node[fbox, right= of tlrc](mar){MyAgent.rudi};
\draw[oarr] (mar.west) -- ++(-.7,0);
\draw[arr] (tlrc.north) ++ (1,0) coordinate (tn) -- node[rlabel]{calls} (rm1.south -| tn);
\draw[arr] (rm1.south) ++ (-1,0) coordinate (rs) -- node[llabel]{imports} (tlrc.north -| rs);

\path (tlrc) ++(0,-.75) node[box, fill=ivory](caai)
  {Agent API Extension (optional)};
\node[rbox, right= of caai]{MyAgentBase.java};
\draw[arr] (caai) -- node[rlabel]{extends} (tlrc);

\path (caai) ++(0,-.75) node[box, fill=darkblue, text=ivory](caid) {Common Agent API Interface Description}
 ++ (0,-.35) node[bbox](caii) {Common Agent API Implementation}
% ++ (0,-.35) node[bbox] {VOnDA Runtime}
 ++ (0,-.45) node[bbox, minimum height=.6cm](cb) {
\begin{minipage}{2cm}\centering Main event\\processing loop\end{minipage}
\begin{minipage}{2cm}\centering RDF Object Access\end{minipage}
}
 ++ (0, -.4) node[bbox] {Dialogue Act Creaton / Comparison}
 ++ (0,-0.35) node[bbox](nlp){
\begin{minipage}{2cm}\centering NLG\end{minipage}
\begin{minipage}{2cm}\centering NLU\end{minipage}
};
\node[rotate=90,yshift=2mm] at (cb.west){\scriptsize VOnDA Runtime};

\node[rbox, right= of caii]{Agent.java};
\node[rbox, right= of caid]{Agent.rudi};
\draw[arr] (caid) -- node[rlabel]{extends} (caai);
\draw (cb) ++(0,.3) -- +(0,-.51);
\draw (nlp) ++(0,.18) -- +(0,-.36);

\path (nlp) ++(0,-.75)
+(-1.125,0) node[fill=lightgray, minimum width=2.25cm, minimum height=.35cm]{}
+(1.125,0) node[fill=darkblue, text=ivory, minimum width=2.25cm, minimum height=.35cm]{}
+(0,0) node[box](ci) {\hspace{2.7ex}Client\color{ivory}\ \ Interface };
\node[rbox, right= of ci]{StubClient.java / MyClient.java};

\draw[arr] (ci.north) ++ (1,0) coordinate (ci1) -- (nlp.south -| ci1);
\draw[arr] (nlp.south) ++ (-1,0) coordinate (nl1) -- (ci.north -| nl1);

\node[minimum width=2cm, minimum height=.7cm, %draw,
      fill=meddarkblue, text=ivory, rotate=90](vc) at (5.3,-.6) {\bf\sffamily VOnDA Compiler};

\node[minimum width=1.3cm, minimum height=.8cm, %draw,
      fill=meddarkblue, text=ivory](dbg) at (7.3,-4.9)
      {\begin{minipage}{1.3cm}\centering\bf\sffamily VOnDA\\Debugger\end{minipage}};

% SUPER TEMPLATE FÜR DIE "TONNE"
\path [fill=midgray, text=black, draw=black] (5.3, -4)
   node[minimum height=1.24cm,minimum width=1.5cm] (db) {\cmp{1.3cm}{RDF\\Database}}
   ++(-.75,0.5)
   -- ++(0,-.9)
   arc [start angle=180, delta angle=180, x radius=.75cm, y radius=1.2mm]
   -- ++(0,.9)
   arc[start angle=0, delta angle=360, x radius=.75cm, y radius=1.2mm];

\draw[arr] (vc) --
  node[above, rotate=90, xshift=1.1mm]{Class \& Predicate}
  node[below, rotate=90, xshift=1.1mm]{Definitions} (db);
\draw[arr, {Stealth}-{Stealth}] (db.west)
       -- node[above]{Belief State Data} (db.west -| cb.east);

\draw[arr](dbg) -- (nlp);
\draw[arr](dbg) -- (db);
\draw[arr](dbg) -- (vc.south west);
\end{tikzpicture}

%%% Local Variables:
%%% mode: latex
%%% TeX-master: "vonda"
%%% End:

  \end{center}
  \caption{Schematic of a \vonda interaction manager implementation}
  \label{fig:architecture}
\end{figure}

A basic \vonda project consists of an ontology, a client interface to
connect the communication channels of the application to the agent,
and a set of rule files that are arranged in a tree, using
\texttt{include} statements. The blue core in
Figure~\ref{fig:architecture} is the run-time system that is part of
\vonda, while all light grey elements are the application specific
parts of the agent. A YAML project file contains all
necessary information for compilation: the ontology, the top-level
rule file and other parameters, like custom compile commands for
\vonda's debugger.

As you can see in the figure, instead of the top-level rule file
directly extending from the framework's abstract \texttt{Agent} class,
you can optionally insert a custom extension of this class and let the
rule agent derive from that. This is meant for special cases when the
Agent class functionality does not suffice or is not exactly the way
you need it (e.g., when more complex synchronisation is needed). Then,
you can use the configuration key \texttt{agentBase} with a fully
specified class name to extend the top-level generated file from your
custom agent base class.

The \vonda compiler translates rule files with the extension \texttt{.rudi} to
Java files, which means that each file is turned into a public class. During
this process, the ontology storing the RDF classes and properties is used to
automatically infer types, resolve whether field accesses are actually accesses
to the database, etc (see section \ref{sec:typeinference}).  Every rule file
can define variables and functions in \vonda syntax which are then available to
all included files, since the generated Java classes create objects for each
included file, and the references are appropriately handled by the compiler.

For the creation of the objects representing the included rule files, there are
two possibilities: in the default case, the object representing the topmost
rule file will be created once at the system start and will remain the same
until the end of the process, which means that all variables defined there will
also be persistent and keep their values, but all other rule file objects will
be recreated during each rule cycle, thus resetting their local variables each
time. By setting the \texttt{persistentVars} configuration flag to
\texttt{true} (or using the \texttt{-p} command line flag), all rule file
objects will be only created once, and all local variables will keep their
values during runtime. This means that the user/programmer is responsible for
resetting them when a rule cycle starts, if that is required.

The current structure assumes that most of the Java functionality that
is used inside the rule files will be provided by the \texttt{Agent}
superclass. There are, however, alternative ways to use other Java
classes directly (see section \ref{sec:javatypes} for further
info). You can either use Java helper classes and/or objects, which is
the preferred way, or you create your own methods and fields in a
custom \texttt{agentBase} and thus make them available to all rule
files. If the compiler does not pick up the type definitions by
itself, you can support it by declaring fields and methods in the type
definition file. In the example of figure ~\ref{fig:architecture},
this would be \texttt{MyAgent.rudi}).

\section{The \vonda Rule Language}
\label{sec:language}
\vonda's rule language looks very similar to Java/C++. There are a number of
specific features which make it much more convenient for the specification of
dialogue strategies. One of the most important features is the way objects in
the RDF store can be used throughout the code: RDF objects and classes can be
treated similarly to those of object oriented programming languages, including
the type inference and inheritance that comes with type hierarchies.

\subsection{The Structure of a \vonda File}

A \vonda file usually consists of a list of (possibly nested) rule statements,
often complemented by variable and function definitions. In this section, we
will describe the elements of the syntax in more detail.

\vonda does not require to group statements in some kind of high-level
structure like e.g. a class. It is, in fact, not possible to define classes in
\texttt{.rudi} files at all, rules and method declarations have to be put
directly into the rule file. The same holds for every kind of valid
(Java-) statement, like assignments, \texttt{for} loops etc. From this, the
compiler will create a Java class where the methods and rules that are
transformed are represented as methods of this specific (generated)
class. All other statements as well as auto-generated calls to the methods
representing the rules will be put into the \texttt{process()} method that
\vonda creates to build a rule evaluation cycle. In doing so, the execution
order of all statements, including the rules, is preserved.

This functionality offers possibilities to e.g. define and process high-level
variables that you might want to have access to in subsequent rules or to insert
termination conditions that prevent some rule executions.

\textbf{Warning:} Variables declared globally in a file will be transformed to
fields of the Java class, as was described above. We found that in very rare
occasions when running the default mode (only the variables of the top-level
rule file keep their values), this can lead to unexpected behaviour when using
them in a propose or timeout block as well as changing them in a global
statement. As proposes and timeouts will not be exceuted immediately, they need
every variable used inside them to be effectively final. \vonda leaves the
evaluation of validness of variables for such blocks to Java. We found that
Java might mistakenly accept variables that are not effectively final, which
might lead to completely unexpected behaviour when proposes and timeouts with
changed variable values are executed.

The globally defined variables and methods defined in the top-level rule file
are always persistent throughout the whole runtime. This is on purpose, and
can be used to define persistent variables also usable in lower-level
rule files, or to be accessed from other java code. If you set the compiler to
the \texttt{persistentVars} mode, this will be true for all variables defined
in the rule files.

\subsection{Rules and rule labels}

The core of \vonda dialogue management are the dialogue rules, which will be
evaluated at run-time system on every trigger generated from the environment or
the internal processing.
A rule (optionally) starts with a name that is given as a Java-like label: an
identifier followed by a colon. Following this label is an
\texttt{if}-statement, with optional \texttt{else} case. The clause of the
\texttt{if}-statement expresses the condition under which the rule, or rather
the \texttt{if} block, is to be executed; in the \texttt{else} block you can
define what should happen if the condition is \texttt{false}, like stopping the
evaluation of (a sub-tree of) the rules if necessary information is missing.

\begin{figure}[htb]
\begin{small}
\begin{lstlisting}
intro:
  if (introduction) {
    is_user_known:
      if (user.unknown) {
        ask_for_name: if (talkative) askForName();
      } else {
        greetUser();
      }
  }
\end{lstlisting}
\end{small}\vspace{-2ex}
\caption{A simple rule}
\end{figure}

Rules can be nested to arbitrary depth, so \texttt{if}-statements inside a rule
body can also be labelled. The labels are a valuable tool for debugging the
system at run-time, as they can be logged live with the debugger GUI
(cf. chapter \ref{sec:debugger}). The debugger can show you which rules were
executed when and what the individual results of each base clause of the
conditions were.

\subsection{The \texttt{propose} and \texttt{timeout} constructs}

There are two statements with a special syntax and semantics: \texttt{propose}
and \texttt{timeout}. \texttt{propose} is \vonda's current way of implementing
probabilistic selection. All (unique) propose blocks that are in active rule
actions are collected, frozen in the execution state in which they were
encountered, like closures known from functional programming languages. When
all possible proposals have been selected, a statistical component decides
on the ``best'', whose closure is then executed.

\begin{figure}[h]
  \centering\small%
\begin{lstlisting}
if (!saidInSession(#Salutation(Meeting)) {
  // Wait 7 secs before taking initiative
  timeout("wait_for_greeting", 7000) {
    if (! receivedInSession(#Greeting(Meeting))
      propose("greet") {
        da = #InitialGreeting(Meeting);
        if (user.name) da.name = user.name;
        emitDA(da);
      }
  }

  if (receivedInSession(#Salutation(Meeting))
    propose("greet_back") { // We assume we know the name by now
      emitDA(#ReturnGreeting(Meeting, name={user.name});
    }
  }
}
\end{lstlisting}\vspace*{-3ex}
  \caption{\label{fig:propose}\texttt{propose} and \texttt{timeout} code example}
\end{figure}

\texttt{timeout}s generate the same kind of closures, but with a different
purpose. They can for example be used to trigger proactive behaviour, or to
check the state of the system after some amount of time, or in regular
intervals. A timeout will only be created if there is no active timeout with
the same name, otherwise, if the time delay is different than that of the last
\texttt{timeout} call, the delay will be set to the new value. For special
needs, the functions in figure~\ref{tbl:timeoutfns} are useful to achieve
specific behaviours based on \texttt{timeout}s.

\begin{figure}[htb]
\begin{tabular}{lp{.65\textwidth}}
\texttt{isTimedOut(\emph{label})}& returns \texttt{true} if a timeout with that
  label fired. This can be reset only by calling
  \texttt{removeTimeout(\emph{label})}, and is especially convenient to
  implement timeouts that should only be triggered once in a session. \\
\texttt{removeTimeout(\emph{label})}& see \texttt{isTimedOut(\emph{label})}\\
\texttt{cancelTimeout(\emph{label})}& cancels an \emph{active} timeout if there
                                      is one, has no effect otherwise \\
\texttt{hasActiveTimeout(\emph{label})}& returns true if there is an
  active timeout with that label \\
\end{tabular}
\caption{\label{tbl:timeoutfns}Functions for fine tuning \texttt{timeout} behaviour}
\end{figure}

There are two variants of \texttt{timeout}: \emph{labeled timeouts}, like the
one in the previous example which run out after the specified time (unless they
are cancelled before running out) and then execute their body, and
\emph{behaviour timeouts}, where the first argument is a dialogue act (see next
section) instead of a label. These are executed either when the specified time
is up or the behaviour that was triggered by the dialogue act is finished
(e.g. the audio generated by a text-to-speech engine ended, or a specified
motion came to an end), whatever comes first.

The following code patterns may help to use the different possibilities that
timeouts offer:

{\small%
\begin{lstlisting}
// timeout triggered exactly once per session
if (! hasActiveTimeout("robot_starts") && ! isTimedOut("robot_starts"))
  timeout("robot_starts", 4000) { ... }

// timeout reoccurring every 1000 milliseconds
if (! hasActiveTimeout("reptimeout"))
  timeout("reptimeout", 1000) { ... }

// ensure that something happens even if the expected condition does not
// become true after 10 seconds
if (! condition && ! hasActiveTimeout("ensure_cond")) {
  timeout("ensure_cond", 10000) {
    if (! condition) {
      // clean up
    }
  }
}
\end{lstlisting}}

\subsection{Stopping Rule Evaluation}
\label{sec:cancelrules}
There are multiple ways to stop rule evaluation locally (i.e. skipping the
evaluation of the current subtree) or globally (i.e. stopping the whole
evaluation cycle).
You can skip the evaluation of a specific rule you are currently in with the
statement \texttt{break label\_name;}. This will only stop the rule with the
respective label (no matter how deep the break statement is nested in it), such
that the next following rule is evaluated next.

If the evaluation is cancelled with the keyword \texttt{cancel}, all of the
following rules in the current file will be skipped (including any included
rule files). If the keyword \texttt{cancel\_all} is used, none of the following
rules, neither local nor higher in the rule tree, will be evaluated. This is
the \vonda way of deciding to not further evaluate whatever triggered the
current evaluation cycle and will mostly be used as an 'emergency exit', as the
dialogue rules should be rejecting any non-matching trigger by themselves.

To leave \texttt{propose} and \texttt{timeout} blocks, you need to use a
\texttt{return} statement without return value, as they are only reduced
representations of normal function bodies.

A detailed description of how the rules of a \vonda project are evaluated will
follow in section~\ref{sec:ruleevaluation}.


\subsection{RDF access and functional vs. relational properties}
\label{sec:rdfaccesses}

\begin{figure}[htb]
\rule{7mm}{0pt}\begin{minipage}{0.45\columnwidth}
\small%
\begin{lstlisting}[numbers=left,numberstyle=\scriptsize]
user = new Animate;
user.name = "Joe";
set_age:
if (user.age <= 0) {
  user.age = 15;
}
\end{lstlisting}
\end{minipage}\vrule\hspace{1ex}
\begin{minipage}{0.44\columnwidth}
    \small\begin{tikzpicture}[
  blob/.style={circle, fill=yellow!50!white, minimum width=2mm},
  txt/.style={node distance=3mm}]
  \draw (0,0) node (agent) [blob]{};
  % BEWARE: RIGHT OF= IS DEPRECATED, DON'T USE IT
  \node (agtxt) [right= 0.1 of agent, txt] {Agent};
  \node (name) [below= 0.4 of agtxt.west, anchor=west]{\emph{name}: \texttt{xsd:string}};
  \node (animate) [blob, below=0.7 of agtxt.west]{};
  \node (antxt) [right= 0.1 of animate, txt] {Animate};
  \node (name) [below= 0.4 of antxt.west, anchor=west]{\emph{age}: \texttt{xsd:int}};
  \node (inanimate) [blob, below= 0.5 of animate, node distance=9mm]{};
  \node (intxt) [right= 0.1 of inanimate, txt] {Inanimate};
  \draw (agent) |- (animate);
  \draw (agent) |- (inanimate);
\end{tikzpicture}
\end{minipage}
  \caption{Ontology and \vonda code}
  \label{fig:rdfobjects}
\end{figure}

Figure \ref{fig:rdfobjects} shows an example of \vonda code, and how it relates
to RDF type and property specifications, schematically presented on the right.
The domain and range definitions of properties are picked up by the compiler
and then used in various places, e.g., to infer types, do automatic code or
data conversions, or create ``intelligent'' boolean tests, like in line 4,
which will expand into two tests, one testing for the existence of the property
for the object, and in case that succeeds, a test if the value is smaller or
equal than zero.

The connection of \vonda to the ontology loaded into HFC during compile time
enables the compiler to recognise the correct RDF class to create a new
instance when creating a new RDF object with \texttt{new}, similar to a Java
object, and to resolve field/property accesses to all RDF instances. Field
accesses as shown in line 2 and 3 of figure \ref{fig:property-access} will be
analysed and transformed into database accesses. Object creation or
assignments, i.e. changes to existing objects, will be immediately reflected in
the database.


\begin{figure}[htbp]
\small\begin{minipage}{0.5\textwidth}
\begin{lstlisting}[numbers=left,numberstyle=\scriptsize]
c = new Child;
nm = c.name;
c.name = "new name";
Set middle = c.middleNames;
c.middleNames += "John";
c.middleNames -= "James";
c.middleNames = null;
\end{lstlisting}
\end{minipage}{\LARGE$\Rightarrow$}\rule{6.4cm}{0pt}\\
\begin{minipage}{0.99\textwidth}
\begin{lstlisting}[numbers=left,numberstyle=\scriptsize]
c = palAgent._proxy.getClass("<dom:Child>").getNewInstance(palAgent.DEFNS);
nm = c.getString("<upper:name>");
c.setValue("<upper:name>", "new name");
middle = c.getValue("<upper:middleNames");
c.getValue("<upper:middleNames>").add("John");
c.getValue("<upper:middleNames>").remove("James");
c.clearValue("<upper:middleNames>");
\end{lstlisting}
\end{minipage}
  \caption{Examples for an RDF property access}
  \label{fig:property-access}
\end{figure}

\vonda will also draw type information from the database. If the name property
of the RDF class \texttt{Child} is of type \texttt{String}, exchanging line 2
by the line \texttt{int name = c.name} will result in a warning of the
compiler. During this process, the compiler will automatically also use the
correspondence of XSD and Java types shown in figure \ref{fig:RdfToJava}.

\begin{figure}[htb]
{\small\ttfamily\begin{center}
\begin{tabular}{ll@{\hspace{8em}}ll}
<xsd:int> & Integer & <xsd:integer> & Long\\
<xsd:string> & String & <xsd:byte> & Byte\\
<xsd:boolean> & Boolean & <xsd:short> & Short\\
<xsd:double> & Double & <xsd:dateTime> & Date\\
<xsd:float> & Float & <xsd:date> & XsdDate\\
<xsd:long> & Long & <xsd:dateTimeStamp> & Long
\end{tabular}\end{center}}\vspace{-2ex}
\caption{\label{fig:RdfToJava}Standard RDF types and the Java types as which they will be recognized}
\end{figure}

If there is a chain of more than one field/property access, every part is
tested for existence in the target code, keeping the source code as concise as
possible (see also figure~\ref{tab:multi-predaccess} in
section~\ref{sec:typeinference}). Also for reasons of brevity, the type of a
new variable needs not be given if it can be inferred from the value assigned
to it.

Moreover, \vonda determines whether an access is made using functional or
relational predicates and will handle it accordingly, assuming a collection
type if necessary. In the rule language, the operators \texttt{+=} and
\texttt{-=} are overloaded. They can be used with sets and lists as shortcuts
for adding and deleting objects. \texttt{a += b} will be compiled to
\texttt{a.add(b)} and \texttt{a -= b} results in \texttt{a.remove(b)}, as shown
in figure~\ref{fig:property-access}.

\paragraph{Creating RDF instances with
  \texttt{new}}\label{sec:new_rdf}

In figure~\ref{fig:property-access}, \texttt{c = new Child;} is used
to create a new RDF instance of class \texttt{Child}. Since every RDF
instance must be part of a namespace, a default namespace string field
\texttt{DEFNS} is defined in the \texttt{Agent} class. Its default
value is \texttt{"def"}, which corresponds to the long URI
\texttt{http://www.dfki.de/lt/default.owl\#}. You have to supply this
name in the initialisation method (\texttt{Agent.init}) of your agent
(it has to be a valid short namespace name, make sure you have the
appropriate namespace mapping in place)\footnote{see also the
  \texttt{init} method of \texttt{ChatAgent} from the example
  project}, and all RDF objects created in the generated \vonda code
will then be created in your custom namespace.

\paragraph{Parameterising field access}\label{sec:field_access_expansion}

To be able to \emph{parameterise} the field access to RDF objects,
\vonda has a special mechanism. Instead of the above
\texttt{c.middleNames}, you could have done the following:

\begin{lstlisting}
mid = "<upper:middleNames>";
Set middle = c.{mid};
\end{lstlisting}

if you use \verb|{<exp>}| in a field access, the compiler assumes
\texttt{exp} to evaluate to a \texttt{String}, and the string
resulting from evaluating \texttt{exp} at runtime will be used as if
you would have specified an identifier with the same name. Be aware
that this only works for access to RDF objects, and that you have to
take care of all type checking and casting by yourself, since the
compiler can not figure out in advance which properties you will access.

\subsection{Casting types in \vonda}
\label{sec:cast}

The syntax for casting expressions explicitely to a specific type is
slightly different than the Java syntax, for reasons of better
readability and easier treatment in parsing the code. \vonda uses the
\texttt{isa} keyword as infix operator, similar to the \texttt{cast()}
in C++, so a Java-style cast \texttt{((Child)c)} will be
\texttt{isa(Child, c)} in \vonda syntax

\subsection{Type inference and overloaded operators}
\label{sec:typeinference}

\vonda allows static type assignments and casting, but in many cases these can
be avoided. If, for example, the type of the expression on the right-hand side
of a declaration assignment is known or inferrable, it is not necessary to
explicitely state it.

You can also declare variables final.

\begin{figure}[htbp]
  \begin{small}
\begin{minipage}{.55\textwidth}
\begin{lstlisting}
if (! c.user.personality.nonchalance){ ... }
\end{lstlisting}
\end{minipage}\rule{2cm}{0pt}{\LARGE$\Rightarrow$}\hfill\\
\begin{lstlisting}
if (!((((c != null) && (c.user != null)) && (c.user.personality != null))
      && (c.user.personality.nonchalance != null))) {
  ...
}
\end{lstlisting}\end{small}\vspace*{-2ex}
\caption{\label{tab:multi-predaccess}Transformation of complex boolean expressions}

\end{figure}
\vspace*{10pt}

A time-saving feature of \vonda which also improves readability is the
automatic completion of boolean expressions in the clauses of \texttt{if},
\texttt{while} and \texttt{for} statements. As it is obvious in these cases
that the result of the expression must be of type boolean. \vonda automatically
fills in a test for existence if it is not. When encountering field accesses,
it makes sure that every partial access is tested for existence (i.e., not
\texttt{null}) to avoid a \texttt{NullPointerException} in the runtime
execution of the generated code.

Be aware that the expansion in the figure only occurs if the multiple field
access is used as boolean test. In the following example, the first clause in
the boolean expression should not be omitted, since a
\texttt{NullPointerException} could still occur because the second clause does
not trigger an automatic test for existence of the \texttt{status} of
\texttt{activity}:

\begin{lstlisting}
if (activity.status && activity.status == "foo"){ ... }
\end{lstlisting}

Many operators are overloaded, especially boolean operators such as
\textbf{\texttt{<=}}, which compares numeric values, but can also be used to
test if an object is of a specific class, for subclass tests between two
classes, and for subsumption of dialogue acts.

\begin{figure}[htbp]
\centering
{\footnotesize%
\begin{minipage}{0.28\textwidth}
\begin{lstlisting}
if (sa <= #Question){
  ...
}
\end{lstlisting}
\end{minipage}\vline\hspace{1em}
\begin{minipage}{0.6\textwidth}
\begin{lstlisting}
if (sa.isSubsumedBy(new DialogueAct("Question")) {
  ...
}
\end{lstlisting}
\end{minipage}}\vspace*{-2ex}
\caption{\label{tab:overloaded-comparison}Overloaded comparison operators}
\end{figure}

\subsection{Dialogue Acts}
\label{sec:caret}

A central functionality of a dialogue system is receiving and emitting dialogue
acts that result from a user utterance resp. can be transformed to natural
language by a generation component to communicate with the user. In \vonda,
the function for sending dialogue acts is called \texttt{emitDA}.

The dialogue act representation is an internal feature of \vonda. We are
currently using the DIT++ dialogue act hierarchy \citep{bunt2012iso} and
shallow frame semantics along the lines of FrameNet
\citep{ruppenhofer2016framenet} to represent dialogue acts. The natural
language understanding and generation units connected to \vonda should
therefore be able to generate or, respectively, process this representation.

\begin{figure}[htb]
  \centering\small\texttt{emitDA(\#Inform(Answer, what=\{solution\}));}
  \vspace*{-1ex}\caption{\label{fig:DA}Dialogue Act Example}
\end{figure}

Figure \ref{fig:DA} shows the dialogue act representation in \vonda, as passed
to, e.g., the \texttt{emitDA} function. \texttt{Inform}\verb|(...)| will be
recognized by \vonda as dialogue act because it has been marked with
\verb|#|. It will then create a new instance of the class DialogueAct that
contains the respective modifications. As a default, arguments of a DialogueAct
creation (i.e., character strings on the left and right of the equal sign) are
seen as and transformed to constant (string) literals, because most of the time
that is what is needed.  Surrounding a character sequence with curly brackets
(\texttt{\{\}}) marks it as an expression that should be evaluated. In fact,
arbitrary expressions are allowed inside the curly brackets, and converted
automatically to a string, if necessary and possible.

While this kind of shallow semantics is enough for many applications, we
already experience its shortcomings when trying to handle, for example, social
talk. One of the next improvements will be the extension of Dialogue Acts to
allow for embedded structures.


\subsection{Declaring External Methods And Fields}
\label{sec:javatypes}

As mentioned before, you can use every method or field you declare in
some Java class or your custom \texttt{Agent} subclass implementation
in your \vonda code. Their declaration in the Java/rudi interface
looks like a normal Java field or method definition
(cfg. figure~\ref{tab:javadef}). It is possible to use generics in
these definitions, although their names are, for complexity reasons,
restricted to one single uppercase letter.

\begin{figure}[htbp]
\small
\begin{lstlisting}
MyType someVariable;          // field of class MyType
MyType someFun(ClA a, ClB b); // method someFun with signature
                              // (ClA, ClB) --> MyType
                              // return type can be void for a procedural method
\end{lstlisting}\vspace*{-2ex}
\caption{\label{tab:javadef}Defintions of existing Java fields and methods for
  \vonda}
\end{figure}

There is a variety of standard Java methods called on Java classes that \vonda
automatically recognises, like e.g. the \texttt{substring} method for
Strings. If you find that you need \vonda to know the signature of a new method
or a field that is defined in some other class that is not your \texttt{Agent}
subclass, you can provide \vonda with knowledge about them by adding their
definition to the interface as follows:

\begin{figure}[htbp]
\centering
\small
\begin{lstlisting}
#SomeClass myType Function(typeA a); // declaration of SomeClass method
#SomeClass myType someVar;           // declaration of SomeClass field
#List<T> T get(int a);               // use of Generics is possible and
                                     // used in type inference
\end{lstlisting}\vspace*{-2ex}
\caption{\label{tab:methoddef}Definition of a non-static method of Java objects}
\end{figure}

It is important to realise that all declarations in the interface are only
compile time information for \vonda and will not be transferred to the compiled
code, whereas declarations in the rule code itself will also appear in the
compiled code.

\subsubsection{Functional constructs}

\vonda allows to specify \texttt{Function} arguments, where lambda
constructions can then be used in the code. Currently, the functions listed in
figure \ref{tab:lambda-functions} are pre-defined in the \texttt{Agent} class.
If you for example want to filter a set of RDF objects by a sub-type relation,
you can write:
{\small\begin{lstlisting}
des = filter(agent.desires, lambda(d) (d <= UrgentDesire));
\end{lstlisting}}
or
{\small\begin{lstlisting}
des = filter(agent.desires, lambda(d) { return (d <= UrgentDesire); });
\end{lstlisting}}

\begin{figure}[htbp]
\small%
\begin{lstlisting}
boolean some(Collection<T> coll, Function<Boolean, T> pred);
boolean all(Collection<T> coll, Function<Boolean, T> pred);
List<T> filter(Collection<T> coll, Function<Boolean, T> pred);
List<T> sort(Collection<T> coll, Function<Integer, T, T> comp);
Collection<T> map(Collection<S> coll, Function<T, S> f);
int count(Collection<T> coll, Function<Boolean, T> pred);
T first(Collection<T> coll, Function<Boolean, T> pred);
\end{lstlisting}\vspace*{-2ex}
\caption{\label{tab:lambda-functions}Functions that take lambda expressions as an argument}
\end{figure}


\subsubsection{Using rules of other rule files with \texttt{include}}

With the \texttt{include} statement, e.g. \texttt{include RuleFile;},
which needs to appear at the root level of rule files, the (compiled)
rules and definitions in \texttt{RuleFile.rudi} and its included files
are added at the position of the \texttt{include}. This is not a
macro-like inclusion, the compiler generates Java classes for every
\texttt{include}.

This inclusion has two important effects. On the one hand, it triggers
the compilation of the included file at exactly this point, such that
any fields and methods known at this time will be available in the
included file. On the other hand, all the rules contained in the
included file will be inserted in the run-time rule cycle at the
specific position of the \texttt{include}, that is, in the resulting
code the \texttt{process()} method of the generated code for the
included file will be executed.

\texttt{include} makes it possible to organize the rules and local
declarations of a project into meaningful sub-units. This supports
modularity, as different subtrees of the \texttt{include} hierarchy can
easily be added, moved, taken away or re-used in different projects.

\subsubsection{Java-Code verbatim in rule files} \label{sec:rudi-verbatim}

To maintain simplicity, \vonda intentionally only provides limited Java
functionalities. Whatever is not feasible in \vonda source code should be done
in methods in the wrapper class or other helper classes.

In cases where this is not possible and you urgently need a
functionality of Java that \vonda cannot parse or represent correctly,
you can use the verbatim inclusion feature. Everything between
\verb|/*@| and \verb|@*/| will be treated like a multi-line Java
comment, meaning the content is not parsed or evaluated further. It
will be transferred to the compiled code as is into the intended
position.

It is strongly discouraged to use this feature extensively, but to use
helper objects/classes instead whereever possible. Not only will code
written this way likely become unreadable if there is too much of it,
it might also not be portable to new \vonda versions, if the way of
generating the Java code changes.

\subsubsection{Java \texttt{import}
  statements} \label{sec:java-import}

You can use \texttt{import} statements as in Java syntax in rule
files, but only at the very beginning. You should however be aware
that \vonda will not know that these classes have been imported, nor
their methods and fields. It will however accept creations of
instances of unknown classes, as well as your casting of results of
unknown methods. If you want \vonda to have type information about
methods called on instances on one of these classes, you can put this
information into your \texttt{typeDef} file (see the beginning of this
section).

\section{The Run-Time System}

The run-time library contains the basic functionality for handling the rule
processing, including the proposals and timeouts, and for the on-line
inspection of the rule evaluation. There is, however, no blueprint for the main
event loop, since that depends heavily on the host application. The run-time
library also contains methods for the creation and modification of shallow
semantic structures (\texttt{DialogueAct}s), and especially for searching the
interaction history for specific utterances. Most of this functionality is
available through the abstract \texttt{Agent} class, which needs to be extended
to a concrete subclass for your application.

There is also functionality to talk directly to the HFC database using queries
(see section \ref{sec:hfc_usage}), in case the object view that was
described in before is not sufficient or too awkward.

\subsection{Rule Evaluation Cycle}
\label{sec:ruleevaluation}

Your \vonda rule files form a tree, starting at the top-level file that you
specify in the configuration file, and the \texttt{include}d rule files. The
evaluation of the rule starts in the top-level files and proceeds in pre-order
through this tree. If you use a \texttt{cancel} or \texttt{cancel\_all}
statement (cf. section~\ref{sec:cancelrules}), the rule evaluation will be
either locally or globally stopped.

The set of your reactive \vonda rules is executed whenever there is a change in
the information state, which is stored in the database. These changes can be
caused by incoming sensor or application data, intents from the speech
recognition, or expired \texttt{timeout}s.  A rule can have direct effects,
like changes in the information state, or system calls. Furthermore, the
\texttt{proposal}s, which are (labeled) blocks of code in a frozen state that
will not be immediately executed, but collected, similar to closures.

All rules are repeatedly applied until a fix point is reached: No new proposals
are generated and there is no change of the information state in the last
iteration. Then, the set of collected proposals is evaluated by a statistical
component, which will select the best alternative. This component can be
exchanged to make it as simple or elaborate as necessary, which also allows to
take into account arbitrary features from the data storage.

At the start of your program, an object of the generated top-level class will
be created which will exist as long as the program is executed. When the
compiler was used in the default mode, temporary objects will be created for
the execution of embedded rules, which cease to exist as soon as all relevant
rules of the subtree have been evaluated. As a consequence, no values created
in these objects will survive the rule evaluation if they are not stored in a
persistent location (e.g., a top-level variable or the database). When the
\texttt{persistentVars} config flag was set to \texttt{true}, the compiler
generates also the objects for embedded rules only on startup and keeps them,
making all local variables permanent, which includes the coder's responsibility
to reset them for a rule evaluation cycle if necessary.

The embedded rules have access to all the variables and methods
declared in higher-level rule files, and all the values produced up to
their call (see also \ref{sec:volatile})

\subsection{Functionality Provided by the Run-Time System}
The following methods are declared in \texttt{src/main/resources/Agent.rudi};
their implementation is provided by Java itself or the \vonda framework.

\vspace*{2ex}

\newcommand{\pgr}[1]{\noindent\textbf{#1}}

\pgr{Pre-added Java methods}
\begin{small}
\begin{lstlisting}
#Object boolean equals(Object e);
#String boolean startsWith(String s);
#String boolean endsWith(String s);
#String String substring(int i);
#String String substring(int begin, int end);
#String boolean isEmpty();
#String int length();

#List<T> T get(int a);
#Collection<T> void add(Object a);
#Collection<T> boolean contains(Object a);
#Collection<T> int size();
#Collection<T> boolean isEmpty();
#Map<S, T> boolean containsKey(S a);
#Map<S, T> T get(S a);
#Array<T> int length;
\end{lstlisting}
\end{small}

\pgr{Short-hand conversion methods from Agent}
\begin{small}
\begin{lstlisting}
int toInt(String s);
float toFloat(String s);
double toDouble(String s);
boolean toBool(String s);
String toStr(T i);  // T in (int, short, byte, float, double, boolean)
\end{lstlisting}
\end{small}

\pgr{Other Agent methods}
\begin{small}
\begin{lstlisting}
// Telling the Agent that something changed
void newData();

String getLanguage();

// Random methods
int random(int limit); // return and int between zero and limit (excluded)
float random();        // return a random float between zero and one (excluded)
T random(Collection<T> coll); // select a random element from the collection

long now();     // return the current time since the epoch in milliseconds

Logger logger;  // Global logger instance

// discarding actions and shutdown
void clearBehavioursAndProposals();
void shutdown();
\end{lstlisting}
\end{small}

\pgr{Timeouts}
\begin{small}
\begin{lstlisting}
void newTimeout(String name, int millis);
boolean isTimedOut(String name);
void removeTimeout(String name);
boolean hasActiveTimeout(String name);
// cancel and remove an active timeout, will not be executed
void cancelTimeout(String name);
\end{lstlisting}
\end{small}

\pgr{Methods dealing with dialogue acts}
\begin{small}
\begin{lstlisting}
// sending of dialogue acts
DialogueAct createEmitDA(DialogueAct da);
DialogueAct emitDA(int delay, DialogueAct da);
DialogueAct emitDA(DialogueAct da);
#DialogueAct String getDialogueActType();
#DialogueAct void setDialogueActType(String dat);
#DialogueAct String getProposition();
#DialogueAct void setProposition(String prop);
#DialogueAct boolean hasSlot(String key);
#DialogueAct String getValue(String key);
#DialogueAct void setValue(String key, String val);
#DialogueAct long getTimeStamp();

// Access to dialogue acts of the current session
// my last outgoing resp. the last incoming dialogue act
DialogueAct myLastDA();
DialogueAct lastDA();

// Did I say something like ta in this session (subsumption)? If so, how many
// utterances back was it? (otherwise, -1 is returned)
int saidInSession(DialogueAct da);
// like saidInSession, only for incoming dialogue acts
int receivedInSession(DialogueAct da);

// Check if we asked a question that is still pending
boolean waitingForResponse();
// Mark last incoming DA as treated and not pending anymore (stop rules firing)
void lastDAprocessed();

DialogueAct addLastDA(DialogueAct newDA);
#DialogueAct void setProposition(String prop);
\end{lstlisting}
\end{small}

\pgr{Functions allowing lambda expressions (functional arguments)}
\begin{small}
\begin{lstlisting}
boolean some(Collection<T> coll, Function<Boolean, T> pred);
boolean all(Collection<T> coll, Function<Boolean, T> pred);
List<T> filter(Collection<T> coll, Function<Boolean, T> pred);
List<T> sort(Collection<T> coll, Function<Integer, T, T> c);
Collection<T> map(Collection<S> coll, Function<T, S> f);
int count(Collection<T> coll, Function<Boolean, T> pred);
T first(Collection<T> coll, Function<Boolean, T> pred);
\end{lstlisting}
\end{small}

\pgr{Methods on \texttt{Rdf} and \texttt{RdfClass} objects}
\begin{small}
\begin{lstlisting}
Rdf toRdf(String uri);
#Rdf String getURI();
#Rdf boolean has(String predicate);
#Rdf long getLastChange(boolean asSubject, boolean asObject);

RdfClass getRdfClass(String s);
boolean exists(Object o);

// return only the name part of an URI (no namespace or angle brackets)
String getUriName(String uri);
\end{lstlisting}
\end{small}


%%% Local Variables:
%%% mode: latex
%%% TeX-master: "userguide"
%%% End:
