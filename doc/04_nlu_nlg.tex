\chapter{Using NLP Modules} \label{sec:external_nlp}

Here¸ we describe the integration of the built-in NLU and NLG modules,
but also the sketch how th integration of other ASR, NLU or NLG
components can be done.

\section{Configuration}
The global configuration file can also contain the configurations for
sub-modules of your system, which need not be part of the core
framework.  In the following, we describe how the SRGS parser and
cplannner are integrated as an example of the plug-in infrastructure
for NLP modules.

In the configuration file, there are tho sections, \texttt{NLU} and
\texttt{NLG}, whose values are passed to the \texttt{LanguageServices}
class, a factory to generate language interpreter and generator
classes. The \texttt{LanguageServices} class and supporting interfaces
and abstract classes, together with the configuration, form a flexible
plug-in framework that also allows to provide your own processing modules.

In these NLU and NLG sections, there are sections for all supported
languages, which is obligatory for \texttt{LanguageServices}; even if
your system only supports one language, you still have to provide the
language key.

The \texttt{LanguageServices} factory uses the language and the
\texttt{class} attribute (mandatory) in the language section to create
a \texttt{NLProcessor} object for the given task. The rest of the
options can be used in the \texttt{init} method of the concrete
processor to set it up accordingly. As was already said in section
\ref{sec:config}, non-\vonda options are allowed in the config file
(no error will result from adding new options at any level) and can be
used by the application, as needed.

The framework has two basic built-in implementations, based on our
extended SRGS implementation and the cplanner graph rewriting
framework, respectively. In addition, as support for the SRGS NLU,
there is a basic tokenizer implementation, which is configured using
th \texttt{tokenizer} section inside the NLU section, in the same way
as the NLU and NLG services.

\section{NLU}
The current built-in NLU is based on our extension of the SRGS
formalism, and uses the following configuration keys:
\texttt{grammar}, \texttt{converter} and \texttt{tokenizer}, which
contains a whole subsection to get a \texttt{Tokenizer} object from
the \texttt{NLProcessor} factory, which will be described later.

\texttt{grammar} points to the root SRGS grammar file in the file
system, relative to the config file's position, and \texttt{converter}
to a cplanner config file that defines a translation the SRGS NLU output
in a declarative way into the internal form (of dialogue acts). The
second part is optional, it is also possible to hard-code that, but
this makes it less flexible, and the current mechanism requires almost
no coding at all (except in cplanner rules).

All NLU classes have to extend the abstract \texttt{Interpreter}
class. You have to implement the
\verb|DialogueAct analyse(String text);| method, and you may want to
supersede the \texttt{init} method. The class already provides a
several convenience functions, e.g., for converting JSON into internal
datastructures, which can then be massaged into the right form using cplanner.

\subsection{Configuration keys for \texttt{TrivialTokenizer}}
The \texttt{TrivialTokenizer} splits the input string using white space as
separators. This is currently the only tokenizer that is available by default
and has the following configuration parameters:
\begin{itemize}
\item \texttt{toLower} (\texttt{boolean}): Convert the string to lower case
  before handing it on
\item \texttt{removePunctuation} (\texttt{boolean}): Remove punctuation
  strings. What is considered to be punctuation is specified in
  \texttt{punctuationRegex}
\item \texttt{punctuationRegex} (\texttt{String}): Must be a valid Java regular
  expression. All matches are removed from the input string.
\end{itemize}

\subsection{Conversion of results using CPlan}

Most NLUs will have fixed formats for the output they generate, which
might not have the desired form or names for keys, etc. This kind of
conversions can be hard-coded, which then results in bigger changes
every time the NLU structures change, or done in a more declarative
form using cplanner\footnote{For a complete documentation, check out
  the gui/doc directory of \url{https://github.com/bkiefer/cplan}}.

While converting list-valued structures is more complex, the usual
moving around and renaming is quite easy, as you can see in the
ChatCat example, which was described in the beginning.

%\subsection{Using NLU connected remotely} % Post 3.0
%MQTT example

\section{NLG}

Text generation currently is implemented using the cplanner graph
rewriting framework (see above), which is a bit of an overkill for
small projects. At time of creation, it was meant to shape the
dialogue acts in a way to create valid input structures for an OpenCCG
generator, which is a technology seldomly used today.

In addition, cplanner provides a quite powerful canned-text
generation, with variables and, e.g., the possibility to compactly
describe morphological variants. Whie it is a quite powerful and maybe
not mastered easily, for larger projects there might be a benefit to
simple dialogue act to string mapping with a hash map at some point.

To implement your own generator, you have to extend the abstract
\texttt{Generator} class, which contains the abstract method

\verb|Pair<String, String> generate(DialogueAct da)|

The first element of the pair is the text to print or send to TTS,
while the second is meant to be a string representation of a robot or
avatar movement to be executed in parallel to the (spoken) text. You
will probably also have to override the \texttt{init} method to
properly set up your component.

%\section{ASR} % Post 3.0
%\section{TTS} % Post 3.0

%%% Local Variables:
%%% mode: latex
%%% TeX-master: "userguide"
%%% End:
